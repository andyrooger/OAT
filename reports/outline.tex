\documentclass{report}
\usepackage{amssymb,amsmath}
\usepackage[mathletters]{ucs}
\usepackage[utf8x]{inputenc}
\usepackage[breaklinks=true,unicode=true]{hyperref}
\setlength{\parindent}{0pt}
\setlength{\parskip}{6pt plus 2pt minus 1pt}
\setcounter{secnumdepth}{0}

\title{Obfuscating Python 3000 - Outline Report}
\author{Andy Gurden\\Supervised by: Herbert Wiklicky}
\date{Friday 11th February}

\begin{document}
\maketitle

\section{Introduction}

With the increasing popularity of interpreted languages like Java or Python, we are starting
to see a change in the way that program code is distributed.

Not so long ago a programmer could write their program in a language such as C. Once compiled
to machine code for a given architecture, the program could be handed out to anyone without
worrying about whether a hacker or reverse-engineer could recover the original source code.
Of course it can be done and there are tools to reverse the translation\cite{cdecomp} but it
is a difficult process and there are methods to make this harder\cite{disres}.

The difference for interpreted languages is that much more of the original code is kept in the
distributed format. For example Java code is usually distributed as bytecode for their virtual
machine. This still holds information about class and method names for example\cite{classinfo},
making it much easier to decompile a Java class.

In a language like Python, it is usually raw source code that is distributed. Therefore a
user need not put in any effort to view the inner workings of the program. While this is good
for the security of a user\cite{noobf}, it is likely the author will not want sensitive
proprietry parts of their code open to inspection or theft. So it is essential that there is a
way to protect it. This is the problem I shall be addressing during this project - specifically
in the Python language (reasons in next section).

It must be pointed out that there are (good and bad) methods to protect Python source code already.
These will also be discussed in the next section along with the reasons I believe my solution can
contribute further.

By the end of this project, I hope to be able to show you a basic functioning but extensible obfuscator
that can modify the control flow of a program with the aim of diverting a human reader from its actual
function. I also hope to show you a sister program that will try to reverse these modifications with the
aim of recovering the original source code.

I shall aim to do this by taking a subset of obfuscation techniques used on other languages (e.g.
\cite{taxobftrans}) and seeing how well these will work in a such a dynamic language as Python. On the
surface this seems rather simple, however due to the nature of Python it can be very difficult to analyse
\cite[p13]{staticanal} in the same way needed for many obfuscation techniques used in statically compiled
languages like Java. These problems may appear or they may not depending on the techinques tried and it
will be the job of this paper to spot them.

\section{Background}

As described in the previous section I shall describe here 

* Obfuscation types (and specific ones for me to try)
* Python code protection (good, bad)
* Language - Python, Why Python 3? (i.e. version, run-time) NASA, google use it, wide.
* Current tools for obfuscation
* Current tools for analysis

MAke it clear I know about important key work in the subject

Enough detail to make self contained, no more.

No routine bullshit

Critically analyse work

Synthesis and analysis

\section{Body}

What I did and why?
Justify tools etc
Key challenges

\section{Evaluation}



\section{Plan of Work}

\section{Introduction}

In this project

Python is a high-level cross-platform interpreted language with a
strong philosophy of readable and elegant code. With its extensive
standard library and ability to be easily extended or embedded into
other applications using native code it has become increasingly
popular among developers.

The fact that Python is interpreted brings with it a number of
advantages but also some drawbacks when it comes to deploying your
software into the real world. For example it means that a Python
program can run anywhere its interpreter will run, which is a great
many places in fact. Unfortunately it also means that an
interpreter is needed on the system before you deploy your code.
This is not so much of a problem as one may expect as the Python
platform is installed by default on most major Linux distributions
and Mac OS. It can also be easily installed on any recent Windows
machine.

Of course there is also a minor speed loss associated with
interpreting the source code before it is run. This is often offset
by much faster development speed and less actual code to achieve
the same goal. In fact by using a just-in-time compiler such as
unladen swallow, the speed difference can be minimised even
further.

The specific problem with using an interpreter that this project
intends to address is that Python needs to be distributed as source
code - in a form that anyone can read. This is likely to be a major
problem for organisations wishing to distribute a system written
entirely in Python as it's quite possible they will not want their
hard work exposed to the world to copy or steal. It may also be the
case that sensitive algorithms inside the program need to be hidden
from the wider public to protect security and keep the software and
its users from abuse.

A number of solutions to this problem exist, all have positives and
negatives and are discussed later. The route I will take during
this project is to create a tool to obfuscate the code, or hide its
meaning from a human reader. This tool will need to be accompanied
by another tool that will try to analyse the code to remove any
modifications made by the obfuscator.

\begin{quote}
You should be able to clearly explain what the problem is, why the
problem is important and why it is difficult to address. You should
also be able to succinctly describe your main idea and what issues
need to be addressed. Also see guidelines on the final report.

Final:

This is one of the most important components of the report. It
should begin with a clear statement of what the project is about so
that the nature and scope of the project can be understood by a lay
reader. It should summarise everything you set out to achieve,
provide a clear summary of the project's background, relevance and
main contributions. It should explain the motivation for the
project (i.e., why the problem is important) and identify the
issues to be addressed (i.e., why the problem is difficult). The
introduction should set the scene for the project and should
provide the reader with a summary of the key things to look out for
in the remainder of the report. When detailing the contributions it
is helpful to provide pointers to the section(s) of the report that
provide the relevant technical details. The introduction itself
should be largely non-technical. It is sometimes useful to state
the main objectives of the project as part of the introduction.
However, avoid the temptation to list low-level objectives one
after another in the introduction and then later, in the evaluation
section (see below), say something like
``All the objectives of the project have been met blah blah\ldots{}''.
A project that meets all its objectives is, by definition, weak and
unambitious. Concentrate instead on the big issues, e.g.~the main
questions (scientific or otherwise) that the project sets out to
answer

\end{quote}
\section{Background}

\subsection{Python}

Python is a cross plaform dynamic language being used by an increasing
number of developers. It is interpreted and so 

\subsection{Code Obfuscation}

Code obfuscation is a technique employed to mask the true function
of a section of code. It involves transforming the source or machine
code should make it difficult for a human to read or understand
while still keeping it functionally equivalent to the original
program. That said, no obfuscation technique is bulletproof and can
always be reverse-engineered given enough time. Therefore the aim of
obfuscating code is just to increase the time and hassle involved in
reverse engineering to an unacceptable level.

These techniques can be employed to protect sensitive algorithms and
intellectual property from copycats or to avoid reverse-engineers
tampering with the program. For example Skype obfuscates their code
(and behaviour) heavily to deter reverse engineers \cite[skypeobf].



Traditionally a language such as C or Java can be obfuscated,
meaning that the source code or bytecode is modified to obscure the
functionality from a human reader. The resulting code may still be
run by a machine. This is particularly important to Java or other
languages that are not compiled to native machine code and Java in
particular can be decompiled fairly easily to retrieve almost the
original code.

While there are many tools to support obfuscation for well
established compiled languages, Python has a great deal fewer
options. This is likely due to a

\begin{quote}
By now you should have studied most of the background work, so the
background section of your outline report should already contain
most of the contents of the background section of your final
report.

Final:

The background section of the report should set the project into
context by relating it to existing published work which you read at
the start of the project when your approach and methods were being
considered. There are usually many ways of solving a given problem,
and you shouldn't just pick one at random. Describe and evaluate as
many alternative approaches as possible. The published work may be
in the form of research papers, articles, text books, technical
manuals, or even existing software or hardware of which you have
had hands-on experience. Your must acknowledge the sources of your
inspiration. You are expected to have seen and thought about other
people's ideas; your contribution will be putting them into
practice in some other context. However, avoid plagiarism: if you
take another person's work as your own and do not cite your sources
of information/inspiration you are being dishonest; in other words
you are cheating. When referring to other pieces of work, cite the
sources where they are referred to or used, rather than just
listing them at the end. Make sure you read and digest the
Department's plagiarism document.

\end{quote}
\begin{quote}
In writing the Background chapter you must demonstrate your
capability of analysis, synthesis and critical judgement. Analysis
is shown by explaining how the proposed solution operates in your
own words as well as its benefits and consequences. Synthesis is
shown through the organisation of your Related Work section and
through identifying and generalising common aspects across
different solutions. Critical judgement is shown by discussing the
limitations of the solutions proposed both in terms of their
disadvantages and limits of applicability.

Typically you can look for Background work using different search
engines including:

\begin{itemize}
\item
  Google Scholar
\item
  IEEExplore
\item
  ACM Digital Library
\item
  Citeseer
\item
  Science Direct
\end{itemize}
\textbf{Note 1}: Often the terms Background, Related Work or State
of the Art are used interchangeably.\\\textbf{Note 2}: Keyword
search is wonderful, but you need the right
Keywords.\\\textbf{Note 2}: IEEExplore, ACM Digital Library and
Science Direct require you to be on the College network to download
the PDF of papers. If at home, use VPN.

\end{quote}
\section{Body}

\begin{quote}
At this stage you should be able to explain the main aspects of the
solution that your project proposes. For more implementation based
projects you should have already the main aspects of the
architecture, the outline of the algorithms, etc. In essence, you
should be able to write a specification of the solution in such
level of detail that if given to an external programmer he should
be able to implement it. Of course, some aspects of the solution
may be missing, in which case you should be able to explain the
main idea for tackling them.

Final:

The central part of the report usually consists of three of four
chapters detailing the technical work undertaken during the
project. The structure of these chapters is highly project
dependent. They can reflect the chronological development of the
project, e.g.~design, implementation, experimentation,
optimisation, evaluation etc. although this is not always the best
approach. However you choose to structure this part of the report,
you should make it clear how you arrived at your chosen approach in
preference to the other alternatives documented in the background.
If you have built a new piece of software you should describe and
justify the design of your program at some high level, possibly
using an approved graphical formalism such as UML. It should also
document any interesting problems with, or features of, your
implementation. Integration and testing are also important to
discuss in some cases. You need to discuss the content of these
sections thoroughly with your supervisor.

\end{quote}
\section{Evaluation}

\begin{quote}
You should be able to explain how you are going to evaluate the
resulting solution of the project in some amount of detail. In
particular you should explain which aspects need to be evaluated
for comparison with existing prior work, what qualitative and
quantitative aspects will need evaluating, which tests and
benchmarks you will run, etc.

Final:

Be warned that many projects fall down through poor evaluation.
Simply building a system and documenting its design and
functionality is not enough to gain top marks. It is extremely
important that you evaluate what you have done both in absolute
terms and in comparison with existing techniques, software,
hardware etc. This might involve quantitative evaluation, for
example based on numerical results, performance etc. or something
more qualitative such as expressibility, functionality, ease-of-use
etc. At some point you should also evaluate the strengths and
weaknesses of what you have done. Avoid statements like "The
project has been a complete success and we have solved all the
problems asssociated with blah\ldots{}; - you will be shot down
immediately! It is important to understand that there is no such
thing as a perfect project. Even the very best pieces of work have
their limitations and you are expected to provide a proper critical
appraisal of what you have done.

\end{quote}
\section{Plan of Work}

\begin{quote}
You must explain what remains to be done in order to complete the
project and roughly what you expect the timetable to be (allowing
sufficient time to write the final report, presentation and work on
the final demonstration). It is not sufficient to simply give a
timetable, you must also explain what the fall-back positions are
if you run out of time and what extensions can be added if you have
more time. You should be able to succinctly discuss the relative
priority of the remaining tasks and its rationale.

\end{quote}

\bibliographystyle{plain}
\bibliography{outline}

\end{document}
