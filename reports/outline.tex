\documentclass{report}
\usepackage{amssymb,amsmath}
\usepackage[mathletters]{ucs}
\usepackage[utf8x]{inputenc}
\usepackage[breaklinks=true,unicode=true,pdfborder={0 0 0},colorlinks=false]{hyperref}
\usepackage{listings}
\lstset{language=Python}
\setlength{\parindent}{0pt}
\setlength{\parskip}{6pt plus 2pt minus 1pt}
\setcounter{secnumdepth}{0}

\title{Obfuscating Python 3000 - Outline Report}
\author{Andy Gurden\\Supervised by: Herbert Wiklicky}
\date{Friday 11th February}

\begin{document}
\maketitle

\section{Introduction}

With the increasing popularity of interpreted languages like Java or Python, we are starting
to see a change in the way that program code is distributed.

Not so long ago a programmer could write their program in a language such as C. Once compiled
to machine code for a given architecture, the program could be handed out to anyone without
worrying about whether a hacker or reverse-engineer could recover the original source code.
Of course it can be done and there are tools to reverse the translation\cite{cdecomp} but it
is a difficult process and there are methods to make this harder\cite{disres}.

The difference for interpreted languages is that much more of the original code is kept in the
distributed format. For example Java code is usually distributed as bytecode for the Java Virtual
Machine (JVM). This still holds information about class and method names for example\cite{classinfo},
making it much easier to decompile a Java class.

In a language like Python, it is usually raw source code that is distributed. Therefore a
user need not put in any effort to view the inner workings of the program. While this is good
for the security of a user\cite{noobf}, it is likely the author will not want sensitive
proprietry parts of their code open to inspection or theft. So it is essential that there is a
way to protect it. This is the problem I shall be addressing during this project - specifically
in the Python language.

It must be pointed out that there are (good and bad) methods to protect Python source code already.
These will also be discussed in the next section along with the reasons I believe my solution can
contribute further.

By the end of this project, I hope to be able to show you a basic functioning but extensible obfuscator
that can modify the control flow of a program with the aim of diverting a human reader from its actual
function. I also hope to show you a sister program that will try to reverse these modifications with the
aim of recovering the original source code.

I shall aim to do this by taking a subset of obfuscation techniques used on other languages (e.g.
\cite{taxobftrans}) and seeing how well these will work in a such a dynamic language as Python. On the
surface this seems rather simple, however due to the nature of Python it can be very difficult to analyse
\cite[p13]{staticanal} in the same way needed for many obfuscation techniques used in statically compiled
languages like Java. Issues with the analysis may appear or they may not, depending on the techiniques
tried and it will be part of this project to find them.

\section{Background}

Here I will talk a little about Python and current solutions for protecting code, as well as some
obfuscation techniques and current tools.

\subsection{Python}

As the introduction states, the project aims to implement known obfuscation techniques on Python source.
The language has widespread use from the popular web framework Django\cite{django} through many applications
from Google\cite{pygoogle} and even embedded as a scripting language for extensions in programs like GNU
Image Manipulation Program (GIMP)\cite{gimp}. Due to its wide and varied use, it is of increased importance
that someone wishing to protect their Python code from theft or reverse-engineering can do. Luckily for me,
by programming in Python I gain a huge standard library full of tools to help with this task. 

To be specific about the code and techniques I am restricting my efforts to using obfuscation methods that will
result in standard Python code so it is portable across any of the many runtime implementations. The version
of Python I am working on is 3.x (nicknamed 3000 or py3k) for both source and generated code. The most popular
version of Python is still 2.x, however this version will soon go stale as the language moves 3.x, on its way
dropping some compatibility with 2.x.

\subsubsection{Methods of Protection}

There are a number of ways Python developers currently try to hide their code. Some of them work and some of
them don't.

For instance there is a misconception that using the Python tool Freeze will protect code. In fact Freeze will
just take your code and compile it to bytecode, zipping it up with the neccessary parts of the runtime to run
on systems without Python installed. As it says in the README\cite{freezereadme} this provides little if any
protection as Python's standard library comes with a disassembler ready to view the bytecode. There are also
tools such as the one mentioned here\cite{pirates} to help analyse the code, as well as a program called
decompyle\cite{decompyle} that will actually try to generate the original source for older bytecode ($\le$version
2.3). There is a similar tool to Freeze, called py2exe\cite{py2exe}, that will create a Windows executable but
this too suffers from the same problems.

From the above paragraph it's easy to spot that another sometimes used method, just distributing the Python
bytecode, also fails to thwart reverse-engineering attempts. In fact there is another problem
with these methods and any others that rely on Python bytecode. Bytecode is an implementation detail\cite{dis}
of the default Python implementation, CPython! By relying on this to distribute your code, you are gluing any
user to a specific implementation on Python and possibly even a specific version. This removes the portability
expected from a Python program and contradicts the condition that solutions must be portable, specified earlier.

A more secure solution for protecting your code, and one that works for many people is to pull the most
sensitive parts out and replace them with C extensions to the Python program. This way the code becomes machine
code, it can use the wealth of obfuscation tools for C and being machine code makes it much
harder to reverse engineer. Again though, this is based on CPython's ability to include C modules, gluing to
user to a particular runtime as well as a particular machine type that you compiled the modules for.

Now a portable way to protect the code is to obfuscate it, performing source to source translations that can
confuse a reader or program analyser. This is not and will never be a perfect solution, however all attempts
to avoid reverse engineering will be overcome eventually. The difference between one method of protection and
another is the time and effort taken to break it and how much the result is worth to the attacker. There are
programs already for obfuscating Python code, though not many. These will be discussed later.

\subsubsection{Ethics of Obfuscating}

Although it's maybe not obvious, there are ethical implications to obfuscation. Specifically for Python, the
language is based on a philosophy of clear and readable code, it actually enforces this to some extend in the
grammer. Obviously obfuscation is an attempt to take this away.

This hits out against certain expectations a user may have about a Python program. A nice part of having your
software handed to you in source form is that you always know what it's doing. Deliberately obscuring the
function of software, but in a format that is almost always open for inspection removes trust. This means
trust in the user is removed by not allowing them to use the software as they wish, but more importantly
trust in the software can be lost.

If the software is hiding what it is doing this could be for a legitimate reason such as hiding novel ideas
from competitors, or it could be to mask malicious content. A user cannot tell, and if the obfuscation is
good enough a program cannot tell which the reason is. This is a technique employed often by malware
writers to avoid detection by anti-virus software\cite{dycodeobf}. In fact it has been argued that any and
all obfuscated code should be treated as if it were malware\cite{noobf}, assuming software is guilty until
proven otherwise.

My view and the view taken for the remainder of the paper is that these issues are for software writers to
resolve. A tool is just a tool, it can be used for research, protecting legitimate software or for hiding
malware. The authors of each would do exactly the same without the tool, it would just take longer.

\subsection{Types of Obfuscation}

Having decided to obfuscate the code from source to source, it is worth taking a look at the types of
obfuscation already used, how they are useful and which I will pursue. I will classify them into 3 categories
taken from here\cite[p10]{desevobf}.

\subsubsection{Layout Obfuscation}

These will apply transformations to the source language or possibly bytecode that do not affect the running of the
program. This is a very common form of obfuscation\cite[p10]{desevobf} and involves things like removing comments,
scrambling identifiers or removing as much whitespace as possible to make the code unreadable.

Comments and identifiers often hold a lot of semantic information as that is what they are designed to do, so
removal of these transformations is quite effective and irreversible. Other transformations involving syntax
can be easily removed by a source formatter and so will only be effective against an impatient human.

While some of these transformations may be easy to perform in traditional languages, there are places where
Pythons constructs can cause problems. Fortunately these kinds of transformations are already covered in existing
tools, although not neccessarily compatible with Python 3. Therefore I will not be dealing much with layout obfuscation.

\subsubsection{Data Obfuscation}

This type of obfuscation transforms data layout and can help to obfuscate the structure of the program. It can be
particularly helpful in Python as the language makes it very easy to inspect programs as they run, dissecting data
structures and learning about how the program works.

Examples of techniques here include adding extra layers of inheritance into a class's inheritance tree or traversing arrays
in unexpected orders.

\subsubsection{Control-flow Obfuscation}

Control flow obfuscation alters or obscures the control flow of the original code. This will confuse an analyser as to the
true control flow of the program. For example a transformation may introduce a conditional branch to dead or broken code
that never happens, however the value of a the condition is hard for the analyser to prove.

Reversal of this type of transformation can require careful analysis and can be difficult to perform. Often the reversal
of this type of transformation will result in optimisation of the code.

These are the type of transformations I intend to implement first.

\subsection{Tools}

I have previously mentioned that there are tools to do similar obfuscation in Python already. Here I will discuss some
examples of these, as well as tools to try to reverse the transformations.

\subsubsection{Python Obfuscation}

There are a number of tools out there that claim to do this. For example BitBoost has a Python obfuscator\cite{bitboost}
that claims to use layout obfuscation as well as "pyschologically inspired techniques" to confuse readers. Sadly as a
single machine license costs \$200 it is not possible to test the tool outside of their web-based demo.

In the free realm, the freeze\cite{freezereadme} or py2exe\cite{py2exe} programs will also obfuscate python very slightly,
though only by distributing bytecode rather than raw source.

Alternatively pyobfuscate\cite{pyobf} will scramble a subset of the identifiers used in your program as well as performing
some other layout transformations. It has it's limitations however\cite{pyobf} so is not powerful enough for real use cases.

Pyobfuscate, along with many of the other tools, hasn't been updated in a long time and so is likely to be unable to cope with
recent versions of Python code.

\subsubsection{Analysis Tools}

For analysis of Python code there are more tools available.

Firstly to reverse the very basic syntax transformations there are a great number of Python pretty printers such as
pygments\cite{pygments} or PythonTidy\cite{pythontidy}. This should never be a particularly difficult task as Python is
designed with readability in mind and enforces clear formatting in the syntax.

PyLint\cite{pylint}, PyChecker\cite{pychecker} and PyFlakes\cite{pyflakes} are all tools designed to help look for possible
bugs in Python code and so may have some use in checking validity of obfuscated source files. They also look for bad design so
could help to determine how difficult a program is to understand for a human reader.

PyDev\cite{pydev} is a plugin for the Eclipse IDE for developing in Python. It performs some useful code analysis on projects to detect
possible bugs and allow easy refactoring of code. While it is implemented in Java as well as Jython (an alternate Python runtime)
it should be possible to use the ideas if needed. Tools for refactoring are easily available as PyDev uses Bicycle Repair
Man\cite{bikerepair}, a Python library for specifically for this task. PyDev also definitely supports Python 3.x.

To attempt decompilation of the Python bytecode if necessary there is a tool called decompyle\cite{decompyle} that has already been
mentioned earlier. This should not be necessary though as we have already discussed reasons to not use bytecode during the project.

Some of the greatest analysis tools come from Python itself. Programs are often run from an interactive Python interpreter, and this
can be used for easy dissection. Python's standard library comes with modules for parsing, assembling and dissassembling source code.
There are also tools for creating interactive sessions within the program. For example by inserting the following code at any point
in the program:

\begin{lstlisting}
from code import InteractiveConsole
InteractiveConsole(locals()).interact()
\end{lstlisting}

You can effectively create a breakpoint and launch an interactive shell to inspect and possibly modify the current local
variables. The program will continue as soon as the shell is closed.

Other more sophisticated tools are available for debugging or inspection such as AntiFreeze\cite{pirates} if necessary,
however Python provides more than enough in its standard library.

\section{Design Overview}

The end product of this project is intended to be a useable source to source Python obfuscator as well as an analytic tool
to assess its effectiveness. If I want to make sure the transformations I make to programs are really correct I need to
know the ins and outs of the language well. This was I can easily pick up on language features that may contradict my
expectations about the behaviour of the code. For example it's easy to naively assume that a simple assignment to an object
variable would be free of side effects. This is not the case, depending on the code the assignment can run absolutely
anything\cite{pyprop} and doesn't even have to assign the given value to a variable.

To get to grips this well with the language, I am using it to write the software. The hope is that heavy use will help to
drill out some of the less obvious features. For the same reason I have chosen to write the code to run on a Python 3.x
interpreter.

Having chosen the language and version I need to design what I'm going to write. This is split into a few basic building
blocks. Specifically the following:

\begin{itemize}
\item A Python parser/lexer to read and make sense of Python source files.
\item Implementations of a number of obfuscating transformations.
\item Implementations of a number of analytic techniques.
\item A Python writer to take my abstract representation of the source code and spit out a file.
\item A user interface!
\end{itemize}

The most fundamental parts here in the workflow are the readers and writers for the source files.
Without these the other parts are useless, so this is where the implementation begins.

...when talking about choices to write the parser, explain why restricting obfuscator to 3.x is ok.

\section{Body}

Essentials
What I did and why?
Justify tools etc
Key challenges

\section{Evaluation}



\section{Plan of Work}

\section{Introduction}

In this project

Python is a high-level cross-platform interpreted language with a
strong philosophy of readable and elegant code. With its extensive
standard library and ability to be easily extended or embedded into
other applications using native code it has become increasingly
popular among developers.

The fact that Python is interpreted brings with it a number of
advantages but also some drawbacks when it comes to deploying your
software into the real world. For example it means that a Python
program can run anywhere its interpreter will run, which is a great
many places in fact. Unfortunately it also means that an
interpreter is needed on the system before you deploy your code.
This is not so much of a problem as one may expect as the Python
platform is installed by default on most major Linux distributions
and Mac OS. It can also be easily installed on any recent Windows
machine.

Of course there is also a minor speed loss associated with
interpreting the source code before it is run. This is often offset
by much faster development speed and less actual code to achieve
the same goal. In fact by using a just-in-time compiler such as
unladen swallow, the speed difference can be minimised even
further.

The specific problem with using an interpreter that this project
intends to address is that Python needs to be distributed as source
code - in a form that anyone can read. This is likely to be a major
problem for organisations wishing to distribute a system written
entirely in Python as it's quite possible they will not want their
hard work exposed to the world to copy or steal. It may also be the
case that sensitive algorithms inside the program need to be hidden
from the wider public to protect security and keep the software and
its users from abuse.

A number of solutions to this problem exist, all have positives and
negatives and are discussed later. The route I will take during
this project is to create a tool to obfuscate the code, or hide its
meaning from a human reader. This tool will need to be accompanied
by another tool that will try to analyse the code to remove any
modifications made by the obfuscator.

\begin{quote}
You should be able to clearly explain what the problem is, why the
problem is important and why it is difficult to address. You should
also be able to succinctly describe your main idea and what issues
need to be addressed. Also see guidelines on the final report.

Final:

This is one of the most important components of the report. It
should begin with a clear statement of what the project is about so
that the nature and scope of the project can be understood by a lay
reader. It should summarise everything you set out to achieve,
provide a clear summary of the project's background, relevance and
main contributions. It should explain the motivation for the
project (i.e., why the problem is important) and identify the
issues to be addressed (i.e., why the problem is difficult). The
introduction should set the scene for the project and should
provide the reader with a summary of the key things to look out for
in the remainder of the report. When detailing the contributions it
is helpful to provide pointers to the section(s) of the report that
provide the relevant technical details. The introduction itself
should be largely non-technical. It is sometimes useful to state
the main objectives of the project as part of the introduction.
However, avoid the temptation to list low-level objectives one
after another in the introduction and then later, in the evaluation
section (see below), say something like
``All the objectives of the project have been met blah blah\ldots{}''.
A project that meets all its objectives is, by definition, weak and
unambitious. Concentrate instead on the big issues, e.g.~the main
questions (scientific or otherwise) that the project sets out to
answer

\end{quote}
\section{Background}

\subsection{Python}

Python is a cross plaform dynamic language being used by an increasing
number of developers. It is interpreted and so 

\subsection{Code Obfuscation}

Code obfuscation is a technique employed to mask the true function
of a section of code. It involves transforming the source or machine
code should make it difficult for a human to read or understand
while still keeping it functionally equivalent to the original
program. That said, no obfuscation technique is bulletproof and can
always be reverse-engineered given enough time. Therefore the aim of
obfuscating code is just to increase the time and hassle involved in
reverse engineering to an unacceptable level.

These techniques can be employed to protect sensitive algorithms and
intellectual property from copycats or to avoid reverse-engineers
tampering with the program. For example Skype obfuscates their code
(and behaviour) heavily to deter reverse engineers.


Traditionally a language such as C or Java can be obfuscated,
meaning that the source code or bytecode is modified to obscure the
functionality from a human reader. The resulting code may still be
run by a machine. This is particularly important to Java or other
languages that are not compiled to native machine code and Java in
particular can be decompiled fairly easily to retrieve almost the
original code.

While there are many tools to support obfuscation for well
established compiled languages, Python has a great deal fewer
options. This is likely due to a

\begin{quote}
By now you should have studied most of the background work, so the
background section of your outline report should already contain
most of the contents of the background section of your final
report.

Final:

The background section of the report should set the project into
context by relating it to existing published work which you read at
the start of the project when your approach and methods were being
considered. There are usually many ways of solving a given problem,
and you shouldn't just pick one at random. Describe and evaluate as
many alternative approaches as possible. The published work may be
in the form of research papers, articles, text books, technical
manuals, or even existing software or hardware of which you have
had hands-on experience. Your must acknowledge the sources of your
inspiration. You are expected to have seen and thought about other
people's ideas; your contribution will be putting them into
practice in some other context. However, avoid plagiarism: if you
take another person's work as your own and do not cite your sources
of information/inspiration you are being dishonest; in other words
you are cheating. When referring to other pieces of work, cite the
sources where they are referred to or used, rather than just
listing them at the end. Make sure you read and digest the
Department's plagiarism document.

\end{quote}
\begin{quote}
In writing the Background chapter you must demonstrate your
capability of analysis, synthesis and critical judgement. Analysis
is shown by explaining how the proposed solution operates in your
own words as well as its benefits and consequences. Synthesis is
shown through the organisation of your Related Work section and
through identifying and generalising common aspects across
different solutions. Critical judgement is shown by discussing the
limitations of the solutions proposed both in terms of their
disadvantages and limits of applicability.

Typically you can look for Background work using different search
engines including:

\begin{itemize}
\item
  Google Scholar
\item
  IEEExplore
\item
  ACM Digital Library
\item
  Citeseer
\item
  Science Direct
\end{itemize}
\textbf{Note 1}: Often the terms Background, Related Work or State
of the Art are used interchangeably.\\\textbf{Note 2}: Keyword
search is wonderful, but you need the right
Keywords.\\\textbf{Note 2}: IEEExplore, ACM Digital Library and
Science Direct require you to be on the College network to download
the PDF of papers. If at home, use VPN.

\end{quote}
\section{Body}

\begin{quote}
At this stage you should be able to explain the main aspects of the
solution that your project proposes. For more implementation based
projects you should have already the main aspects of the
architecture, the outline of the algorithms, etc. In essence, you
should be able to write a specification of the solution in such
level of detail that if given to an external programmer he should
be able to implement it. Of course, some aspects of the solution
may be missing, in which case you should be able to explain the
main idea for tackling them.

Final:

The central part of the report usually consists of three of four
chapters detailing the technical work undertaken during the
project. The structure of these chapters is highly project
dependent. They can reflect the chronological development of the
project, e.g.~design, implementation, experimentation,
optimisation, evaluation etc. although this is not always the best
approach. However you choose to structure this part of the report,
you should make it clear how you arrived at your chosen approach in
preference to the other alternatives documented in the background.
If you have built a new piece of software you should describe and
justify the design of your program at some high level, possibly
using an approved graphical formalism such as UML. It should also
document any interesting problems with, or features of, your
implementation. Integration and testing are also important to
discuss in some cases. You need to discuss the content of these
sections thoroughly with your supervisor.

\end{quote}
\section{Evaluation}

\begin{quote}
You should be able to explain how you are going to evaluate the
resulting solution of the project in some amount of detail. In
particular you should explain which aspects need to be evaluated
for comparison with existing prior work, what qualitative and
quantitative aspects will need evaluating, which tests and
benchmarks you will run, etc.

Final:

Be warned that many projects fall down through poor evaluation.
Simply building a system and documenting its design and
functionality is not enough to gain top marks. It is extremely
important that you evaluate what you have done both in absolute
terms and in comparison with existing techniques, software,
hardware etc. This might involve quantitative evaluation, for
example based on numerical results, performance etc. or something
more qualitative such as expressibility, functionality, ease-of-use
etc. At some point you should also evaluate the strengths and
weaknesses of what you have done. Avoid statements like "The
project has been a complete success and we have solved all the
problems asssociated with blah\ldots{}; - you will be shot down
immediately! It is important to understand that there is no such
thing as a perfect project. Even the very best pieces of work have
their limitations and you are expected to provide a proper critical
appraisal of what you have done.

\end{quote}
\section{Plan of Work}

\begin{quote}
You must explain what remains to be done in order to complete the
project and roughly what you expect the timetable to be (allowing
sufficient time to write the final report, presentation and work on
the final demonstration). It is not sufficient to simply give a
timetable, you must also explain what the fall-back positions are
if you run out of time and what extensions can be added if you have
more time. You should be able to succinctly discuss the relative
priority of the remaining tasks and its rationale.

\end{quote}

\bibliographystyle{plain}
\bibliography{outline}

\end{document}
