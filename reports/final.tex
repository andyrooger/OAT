\documentclass{report}
\usepackage{amssymb,amsmath}
\usepackage[mathletters]{ucs}
\usepackage[utf8x]{inputenc}
\usepackage[breaklinks=true,unicode=true,pdfborder={0 0 0},colorlinks=false]{hyperref}
\usepackage{listings}
\lstdefinelanguage{obftool}{morekeywords={help,quit,status,parse,explore,format}}
\lstset{language=Python, numbers=left, showstringspaces=false, frame=single}
\usepackage{pst-gantt}
\setlength{\parindent}{0pt}
\setlength{\parskip}{6pt plus 2pt minus 1pt}
\setcounter{secnumdepth}{0}

% For title page
\newcommand{\HRule}{\rule{\linewidth}{0.5mm}}

\begin{document}
\begin{titlepage}

\begin{center}

% Upper part of the page
\textsc{\LARGE Imperial College London}\\[1.5cm]
\textsc{\Large Undergraduate Individual Project}\\[0.5cm]

% Title
\HRule \\[0.4cm]
{\huge \bfseries Obfuscating Python 3000} \\[0.4cm]
\HRule \\[0.4cm]

Code on git at: ssh://user@shell4.doc.ic.ac.uk/homes/asg08/git/ugproject.git \\[1.5cm]

% Author and supervisor
\begin{minipage}{0.4\textwidth}
\begin{flushleft} \large
\emph{Author:}\\
Andy Gurden
\end{flushleft}
\end{minipage}
\begin{minipage}{0.4\textwidth}
\begin{flushright} \large
\emph{Supervisor:} \\
Herbert Wikliky
\end{flushright}
\end{minipage}

\vfill

% Bottom of the page
{\large \today}

\end{center}
\end{titlepage}

\begin{abstract}
TODO write abstract!
\end{abstract}

\section*{Acknowledgements}

TODO - write acknowledements - Hayley Hannam, Herbert Wikliky, Chris Hankin, Alex Lamaison

\tableofcontents

\clearpage

\section{Introduction}

With the increasing popularity of interpreted languages like Java or Python, we are starting
to see a change in the way that program code is distributed.

Not so long ago, a programmer could write their program in a compiled language such as C. This would then be compiled
to machine code for a given architecture, and the program could be handed out to anyone without
worrying about whether a hacker or reverse-engineer could recover the original source code.
Of course, it can be done and there are tools to reverse the translation called decompilers\cite{cdecomp} but it
is a difficult process and there are methods to make this harder \cite{disres}.

The difference for interpreted languages is that much more of the original code is kept in the
distributed format. Java code, for example, is usually distributed as bytecode for the Java Virtual
Machine (JVM). This still holds information about high level constructs such as class and method names for example \cite{classinfo},
making it much easier to decompile a Java class.

In a language like Python it is usually raw source code that is distributed, therefore a
user need not put in any effort to view the inner workings of the program. While this is good
for the security of a user \cite{noobf}, it is often the case that the author will not want sensitive,
perhaps proprietary, parts of their code open to inspection or theft. With this in mind, we will work on ways to protect it.

We should note that there are existing solutions for the protection of Python source code, examples of which vary in quality.
These will be discussed in the next section, along with their drawbacks and areas to which we can contribute further.

The approach we will take to the above problem is to obfuscate the Python source code. This technique involves changing the
original program in some way so that it is more confusing to a reader or more difficult to analyse while still being correct.

More specifically, during this project we shall implement a basic functioning, but extensible, obfuscator
that can modify the control flow of a program with the aim of diverting a human reader from its actual
function. This will act upon Python source code and should help to address the issue described above.
In the interest of evaluation, we will also create a sister program that will try to reverse the modifications performed by the
obfuscator with the aim of recovering the original source code. 
We shall aim to create these tools by taking a subset of obfuscation techniques used on other languages (e.g.
loop flattening or instruction reordering \cite{taxobftrans}) and looking at how well these will transfer to such a dynamic language as Python.

On the surface this may appear rather simple, however as Python is such a highly dynamic language it can be very difficult to analyse
\cite[p13]{staticanal} as thoroughly as a more traditional compiled language such as Java. In such languages, it is much easier to perform a static
analysis of the code. This is simply not feasible for many applications in Python, yet many program transformations will require this type of
thorough analysis. Issues in this area may appear or they may not, depending on the techniques we decide to implement, and it will be part of
this project to notice and evaluate these issues.

\section{Background}

Here we will discuss Python and some solutions for protecting code, as well as some obfuscation techniques
and current tools.

\subsection{Python}

As we have discussed in the introduction, the project aims to implement known obfuscation techniques on Python source.
The language is used widely, from the popular web framework Django \cite{django} through many applications
from Google \cite{pygoogle} and even embedded as a scripting language for extensions in programs like GNU
Image Manipulation Program (GIMP) \cite{gimp}. Due to its wide and varied use, it is of increased importance
that someone wishing to protect their Python code from theft or reverse-engineering can do so.

To be specific about the code and techniques, we will restrict our efforts to using obfuscation methods that will
result in standard Python code so that it is portable across any of the many Python runtime implementations. The version
of Python we will be working with is 3.x (nicknamed 3000 or py3k) for both source and generated code. The most popular
version of Python is still 2.x, however this version will soon ``become stale'' as the language moves towards 3.x, on its way
dropping some compatibility with 2.x.

\subsubsection{Methods of Protection}

There are a number of ways Python developers currently try to hide or protect their code, with varying degrees
of success and security.

One method is to use a Python tool called Freeze. This tool will simply take given source code and compile
it to bytecode. This will be combined with necessary parts of the runtime to allow a program to run
on systems without Python installed. As it says in the README \cite{freezereadme} this provides little if any
protection as Python's standard library comes with a disassembler ready to view the bytecode. There are also
tools such as AntiFreeze \cite{pirates} to help analyse the code, as well as a program called
decompyle \cite{decompyle} that will actually try to generate the original source for older bytecode ($\le$version
2.3). There is a similar tool to Freeze, called py2exe \cite{py2exe}, that will create a Windows executable but
this too suffers from the same problems.

From the above paragraph, it's easy to spot that just distributing the Python bytecode, another method occasionally
used, also fails to thwart reverse-engineering attempts. In fact there is another problem
with these methods and any others that rely on Python bytecode. Bytecode is an implementation detail \cite{dis}
of the default Python implementation, CPython. Relying on this to distribute code would glue any
user to a specific implementation of Python and possibly even to a specific version. This would remove the portability
expected from a Python program and contradict the condition that solutions must be portable, specified earlier.

A more secure solution for protecting Python code, and one that works for many people, is to remove the most
sensitive parts and replace them with C extensions to the Python program. This way the code becomes machine
code, it can use the wealth of obfuscation tools for C and being compiled machine code makes it much
harder to reverse engineer. Again though, this is based on CPython's ability to include C modules, restricting a
user to a particular runtime as well as a particular machine architecture (that the modules are compiled for).

A portable way to protect the code would be to obfuscate it, performing source to source translations that can
confuse a reader or program analyser. This is not and will never be a perfect solution, because all attempts
to avoid reverse engineering will be overcome eventually. The difference between one method of protection and
another is the time and effort taken to break it and how much the result is worth to the attacker. There are already
programs for obfuscating Python code, though the selection is limited. These will be discussed later.

\subsubsection{Ethics of Obfuscating}

Although it may not be obvious, there are ethical and possibly even legal implications to obfuscation.
Specifically for Python, the language is based on a philosophy of clear and readable code, and actually
enforces this by the design of the grammar. Obviously, obfuscation is an attempt to take this away.

This hits out against certain expectations a user may have about a Python program. A key benefit of having your
software handed to you in source form is that you always know what it's doing. Deliberately obscuring the
function of software, but in a format that is almost always open for inspection, removes trust. The vendor or developer is
denying the user trust by not allowing them to use the software as they wish, but more importantly the user's
trust in the software can be lost.

If the software is hiding what it is doing this could be for a legitimate reason, such as hiding novel ideas
from competitors, or it could be to mask malicious content. A user or program, provided the obfuscation is
good enough, cannot tell which the reason is. This is a technique often employed by malware
writers to avoid detection by anti-virus software \cite{dycodeobf}. In fact, it has been argued that any and
all obfuscated code should be treated as if it were malware \cite{noobf}, assuming software is guilty until
proven otherwise.

My view and the view we will take for the remainder of the paper is that these issues create more reason for
building a tool such as this. By allowing issues such as those described above to block research into a subject
we are making ourselves much more vulnerable to those who would abuse the technology. Those who may abuse any
product of this project will be free to do so, as will those who would use the tools and information legitimately.

It is important to note that although my conclusion on this subject would be the same were I based outside of
the UK, the legal implications stemming from it may not be. For example in German law, clause 202(c) \cite{202c}
could make it illegal to write or distribute this software if it could be used for
certain forms of cybercrime. As discussed above, it probably could, and would have implications for anyone distributing
or using the information presented here.

\subsection{Types of Obfuscation}

Having decided to obfuscate the code from source to source, it is worth taking a look at the types of
obfuscation already used, how they are useful and which we will pursue. We will classify them into 3 categories
suggested by \cite[p10]{desevobf}.

\subsubsection{Layout Obfuscation}

These will apply transformations to the source language or possibly bytecode that do not affect the running of the
program. This is a very common form of obfuscation \cite[p10]{desevobf}, and involves transformations such as removing comments,
scrambling identifiers, or removing as much whitespace as possible to make the code unreadable.

Comments and identifiers often hold a lot of semantic information as they are designed to do.
Removal of this information by deletion or renaming is quite effective and almost always irreversible. Other transformations involving syntax
can be easily removed by a source formatter and so will only be effective against an impatient human.

While some of these transformations may be easy to perform in traditional languages, there are places in which
Python's constructs can cause problems. Fortunately these kinds of transformations are already covered in existing
tools, although these are not necessarily compatible with Python 3. For these reasons, we will be focusing on other forms of
obfuscation which have received less attention.

\subsubsection{Data Obfuscation}

This type of obfuscation transforms data layout and can help to obfuscate the structure of the program. It can be
particularly helpful in Python as the language makes it very easy to inspect programs as they run, dissecting data
structures and learning about how the program works.

Examples of techniques here include adding extra layers of inheritance into a class's inheritance tree or traversing arrays
in unexpected orders.

\subsubsection{Control-flow Obfuscation}

Control flow obfuscation alters or obscures the control flow of the original code. This will confuse an analyser as to the
true control flow of the program. For example, a transformation may introduce a conditional branch to dead or broken code
that never happens, however the value of this condition is hard for the analyser to prove or evaluate before runtime.

Reversal of this type of transformation can require careful analysis and can be difficult to perform. Often the reversal
of this type of transformation could result in optimisation of the code.

These are the type of transformations we will attempt to implement first.

\subsection{Tools}

We have previously seen that there are tools to do similar obfuscation in Python already. We will now discuss some
examples of these, as well as other tools to try to reverse the transformations.

\subsubsection{Python Obfuscation}

There are a number of tools out there that claim to do this. For example BitBoost Systems have a Python obfuscator \cite{bitboost}
that claims to use layout obfuscation as well as ``psychologically inspired techniques'' to confuse readers. Sadly, as a
single machine license costs \$200 it is not possible to test the tool outside of their web-based demo.

In the free realm, the freeze \cite{freezereadme} or py2exe \cite{py2exe} programs will also obfuscate python very slightly,
though only by distributing bytecode rather than raw source.

Alternatively pyobfuscate \cite{pyobf} will scramble a subset of the identifiers used in programs as well as performing
some other layout transformations. It has its limitations however \cite{pyobf}, so is not powerful enough for real use cases.

Pyobfuscate, along with many of the other tools, has not been updated in a long time and so is likely to be unable to cope with
recent versions of Python code.

\subsubsection{Analytic Tools}

For analysis of Python code there are more tools available.

Firstly to reverse the very basic syntax transformations there are a great number of Python pretty printers such as
pygments \cite{pygments} or PythonTidy \cite{pythontidy}. This should never be a particularly difficult task as Python is
designed with readability in mind and enforces clear formatting in its syntax.

PyLint \cite{pylint}, PyChecker \cite{pychecker} and PyFlakes \cite{pyflakes} are all tools designed to help look for possible
bugs in Python code and so may have some use in checking validity of obfuscated source files. They also look for bad design and so
could help to determine how difficult a program is to understand for a human reader.

PyDev \cite{pydev} is a plugin for the Eclipse IDE for developing in Python. It performs some useful code analysis on projects to detect
possible bugs and allow easy refactoring of code. While it is implemented in Java as well as Jython (an alternate Python runtime)
it should be possible to reuse the ideas if needed. Tools for refactoring are easily available as PyDev uses Bicycle Repair
Man \cite{bikerepair}, a Python library designed specifically for this task. PyDev also definitely supports Python 3.x as opposed to the
other tools in which this is unclear.

To attempt decompilation of the Python bytecode if necessary, there is a tool called decompyle \cite{decompyle} that has already been
mentioned. This should not be necessary though as we have already discussed reasons not to use bytecode during the project.

Some of the greatest  tools for analysis come from Python itself. Programs are often run from an interactive Python interpreter, and this
can be used for easy dissection. Python's standard library comes with modules for parsing, assembling and disassembling source code.
There are also tools for creating interactive sessions within the program. For example by inserting the following code at any point
in the program:

\begin{lstlisting}
from code import InteractiveConsole
InteractiveConsole(locals()).interact()
\end{lstlisting}

We can effectively create a breakpoint and launch an interactive shell to inspect and possibly modify the current local
variables. The program will continue as soon as the shell is closed.

Other more sophisticated tools are available for debugging or inspection such as AntiFreeze \cite{pirates} if necessary,
however Python provides more than enough in its standard library.

\section{Creating the Software}

The end product of this project is intended to be a usable source to source Python obfuscator as well as an analytic tool
to assess its effectiveness. If we are to make sure the transformations we make to programs are really correct, we should
know the ins and outs of the language well. This way we can easily pick up on language features that may contradict our
expectations about the behaviour of the code. For example it's easy to naively assume that a simple assignment to an object
variable would be free of side effects. This is not the case, depending on the code the assignment can run absolutely
anything \cite{pyprop} and doesn't even have to assign the given value to a variable.

To learn the language to the desired degree we will also use it to write our software. The hope being that heavy use will help to
drill out some of the less obvious features. For the same reason, we will be writing the code specifically to run on a Python version 3.x
interpreter.

Having chosen the language and version we will use to develop in, we need to design the software we are about to write. This is split into a few basic building
blocks. Specifically the following:

\begin{itemize}
\item A Python parser/lexer to read and make sense of Python source files.
\item Implementations of a number of obfuscating transformations.
\item Implementations of a number of analytic techniques.
\item A Python writer to take my abstract representation of the source code and output a file.
\item A user interface.
\end{itemize}

The most fundamental parts here in the workflow are the readers and writers for the source files.
Without these the other parts are useless, so this is where the implementation will begin.

\subsection{Reading and Writing Source}

There are plenty of free tools that read and write source as part of their tasks. For example PyDev, mentioned in the background section, will
perform these tasks during code formatting. For this project we have the choice between writing software from scratch, attempting
to recycle relevant parts from free projects, or using libraries to take the work away from ourselves.

The first option we should look at is to write the code from scratch. This could be very time consuming, requiring
us to write complicated software as well as studying documentation on Python's grammar which is freely available online \cite{pygrammar}.
While the language is well documented it does seem wasteful to repeat work that has been done in so many other places.

Next we look at extracting the functionality from other free (open source) software. The most fitting source to source translation tools
we could use for this are PyDev or the 2to3 tool \cite{2to3}, used to convert Python 2.x source code to Python 3.x. Ultimately we will not
delve into these tools due to the amount of time we could squander searching for relevant parts of code. It's likely we
would come to the conclusion that they outsource most of the work to separate libraries anyway.

This leaves us using libraries. In fact the 2to3 documentation points to a Python standard library called lib2to3 that can be used to
perform automatic translation on Python source. Unfortunately further reading shows that the library API is unstable and could change by
the next release. This is not something we want to have to deal with.

Fortunately the Python standard library contains a number of modules devoted to its own language. By using these to parse a source file
we can hold the source as an abstract syntax tree or AST. This will give us the freedom to transform the language as we like and additionally the only code we
will need to write would be for the AST translations and an AST to source writer.

There are some flaws to storing the program as an AST as this drops certain information.
For example, the elif statement, meaning the same as an else if statement in other languages allows many tests without additional levels of nesting in source.
When this construct is parsed it becomes an if statement within an else statement and can create large changes to the expected code and level of nesting. Information
such as comments are also dropped from the representation. These seem to be reasonable compromises however and so this is the option we will use. It gives
a lot of freedom, whilst being fast to write.

Having made these decisions we can describe a solid design and implementation for the AST writer. The design is fairly simple.
At the centre lies an abstract class called SourceWriter that any code wishing to write source may request. Derived from this
is a very basic concrete source writer called BasicWriter. This class recursively writes out the tree in the
most basic form possible. Every node that could exist in the tree has a method used to write itself to source, and these can be overridden by any
other derived classes to change the look of that particular node. In addition to these methods there are some hooks to determine how
the source looks as a whole. These affect things such as the level of indentation or the way in which a new statement begins.

This makes the source writer easily extendable to create pretty printers, minifiers\footnote{Minification of a program involves shrinking its file size by removing as many unnecessary characters as possible. It is often used for interpreted languages such as JavaScript that need to be lightweight for transmission.}
or, even at a later date if we choose to extend our implementation, a basic layout obfuscator. In fact we will also produce a PrettyWriter class derived from BasicWriter
as an example, designed to produce readable code. PrettyWriter will alter the printing of a number of nodes, specifically it:

\begin{itemize}
\item Checks if docstrings span multiple lines and if so writes them as such.
\item Writes function arguments on multiple lines when the list is expected to be complicated. Actually this looks for arguments with annotations and splits the entire function's argument list if one is found.
\item Inserts blank lines between certain types of statements to guess and separate logical blocks of code.
\end{itemize}

Using this writer should allow us to concentrate on the effect of obfuscations later, rather than struggling to read the generated code. To illustrate, the
following is a sample program passed through our implementation of BasicWriter and PrettyWriter. We can see a large difference between the source code
returned by the two writers, one is much uglier than the other.

Sample Program

\lstinputlisting[basicstyle=\tiny]{../../example/example.py}

Output from BasicWriter

\lstinputlisting[basicstyle=\tiny]{../../example/example.py.basicformat}

Output from PrettyWriter

\lstinputlisting[basicstyle=\tiny]{../../example/example.py.prettyformat}

All of these writers will be held in the writer package.

\subsection{User Interface}

The next important part to think about is the user interface. This tool would be difficult to use without one.

Current tools such as pyobfuscate or BitBoost's obfuscator are automated. This means while a user does have a little control
over what happens, they really just insert a file and are returned an unreadable one. We want to be able to effectively
evaluate what our tool is doing to the source code, so we need control over what gets changed and which transformations are
performed. Therefore this is not really an approach we can use.

Knowing we have to be able to direct and control transformations, there are a few different ideas of how. In general, these fit into
either solutions that involve us interactively looking through the AST and choosing how to transform or analyse certain
nodes, or those that let us mark pieces of the code for obfuscation.

There are methods already used in other tools that we could use to mark code. For example the doctest module in the standard
library uses docstrings (programmatically accessible comments) to specify unit tests for the code. Another possibility would
be to use empty decorators to mark functions or classes for transformation. These are usually used to transform the code in
some way at run time, but could easily do nothing and be picked up as flags for the obfuscator.

The trouble with these methods is that transformations would be restricted to single nodes that support either docstrings or
decorators. For the fine control we need when evaluating the product, we will actually use an interactive explorer.
Transformations or analysis may then be performed at which ever part of the code the user is currently pointing to.

This brings a new question, one of how the explorer will look. The two main choices are a command line tool or a GUI. Again
Python contains tools to make both much easier. These come in the form of libraries such as the cmd library \cite{pycmd} for command
interpreters or Tkinter \cite{pytkinter} for GUIs.

In the interest of speed of development, and as user friendliness is not the main focus for us, we will use a command interface.
More specifically we will implement an interactive interpreter designed to be similar to to a generic Unix shell.

Both the obfuscation and analytic parts of the software will be accessible from this interface, and so it will be named
oat (Obfuscation and Analysis Tool). As well making it easier for us to refer to the inteface, we have also opened the possibility
of creating a graphical interface to the toolset named goat.

In our implementation, the standard Python command interface offering has been extended to make it easy to plug new commands in as
objects, allowing easy extensions with new analytic techniques or new obfuscating transformations. Obviously the
interface is worthless without actually plugging in some commands. Before we start to look at the obfuscation and analysis of code, the set of commands
available from the interface is as follows:

\paragraph{Help}

This will list help for the program as a whole as well as individual commands. It is an essential tool for a user to find their way around a
command based program such as this one.

\paragraph{Quit}

Used to leave the program. Also very necessary as this is not built in to the default interpreter.

\paragraph{Status}

This will send a message to all other commands asking them to relay their status if they have anything to contribute. It can be
used to check the state of certain parts of the program.

\paragraph{Parse}

Actual parsing of Python source files is done using this command. It uses the Python library functions to do the parsing but also
deals with problems like finding the files and handling incorrect syntax. Once a correct source file is parsed, the resulting AST
is stored in memory for use by the other commands. It is also possible to save and load a previously edited AST, so long as the original
source file is retained.

\paragraph{Explore}

This is used to explore an AST generated by the parse command. It will always point to a particular node in the tree and allows
inspection of any of the nodes properties. It is also possible to traverse the tree, moving between any of the current node's parents or
children freely.

\paragraph{Format}

The integration point for the source writers. This can print either the whole abstract syntax tree or the subtree pointed to by
explore. The printing style can also be changed between basic or pretty, and possibly other styles if the writers for these are created in future.
If the user chooses, output from this command may be written to disk.

\paragraph{Mark and Reorder}

Will be introduced later.

Where possible all of these commands implement auto-completion to guide the user and save time.

The following is an example session:

\begin{lstlisting}[language=obftool,basicstyle=\small]
andyrooger@DellBoy:~/git/ugproject/src$ ./quickstart.py 
Welcome to this little obfuscation tool.
If you're confused, type help!
--) help

Documented commands (type help <topic>):
========================================
help  quit  status  parse  explore  format


--) help parse
usage: parse file

Parse a source file to an AST.

positional arguments:
  file  File to parse.

--) parse 
interactive    quickstart.py  thirdparty     writer         
--) parse quickstart.py

--) status
Parsed tree: <_ast.Module object at 0x1bdab90> : quickstart.py
Viewing node: Module

--) explore
Looking at: Module

Fields:
  body - Block of statements

--) explore body
Looking at: Block of statements

--) explore
Looking at: Block of statements

Fields:
  0 - Expr (line 7, col -1)
  1 - ImportFrom (line 9, col 0)
  2 - FunctionDef (line 11, col 0)
  3 - If (line 19, col 0)

--) explore 3
Looking at: If (line 19, col 0)

--) explore
Looking at: If (line 19, col 0)

Fields:
  test - Compare (line 19, col 3)
  body - Block of statements
  orelse - Block of statements

--) explore -a
Looking at: If (line 19, col 0)

Attributes:
  lineno - 19
  col_offset - 0

Fields:
  test - Compare (line 19, col 3)
  body - Block of statements
  orelse - Block of statements

--) format -f
if __name__ == '__main__':
    go()

--) quit
\end{lstlisting}

\section{Reordering Instructions}

In the previous section we saw the basic blocks of oat fall into place. The code
is already set to read, inspect, and write out Python source. Now we can move
on to fulfilment of the project aims; to analyse and obfuscate the code.

The first technique we will seek to implement is the reordering of instructions. As it is required that
any implemented obfuscating transformations have an accompanying analysis that can be used to (attempt to)
reverse the modification, we will also try to implement this.

\subsection{Aims}

Usually a (well-written) program is ordered in such a way as to make it easy for the reader to follow.

There are two aspects of the layout we could inspect at here; the order of statements, and their grouping (i.e. blocks of statements
separated, for example, by blank lines). More often than not, the grouping (in a well written program) indicates
to the reader that all statements in a single group are aimed towards achieving a common goal.
Knowing this, it is easier for a human reader to read the program with a higher level understanding
of its function.

Likewise, the ordering of statements in a program often guides a reader through a human thought process. There may
be many other valid and equally correct orders for the statements, however the program will tend to be written in
an order most easy for the programmer to follow. This in turn tends to make it easier for the reverse engineer to follow.

We need not worry about breaking up the groups, these are lost during conversion from source to abstract syntax tree.
Therefore we will only be dealing in this section with reordering the statements in our programs.

\subsection{Effects}

As stated above, the tendency to insert this extra information in the order of statements is human in nature. To a machine the
order of instructions, if they are still correct, will not usually make a difference to a programatic analysis. We still have an
equivalent program as we have not removed or added any extra steps into the code, only changed the order.

This means it is mainly human analysis and understanding we will be hindering by removing this information. However good
our obfuscation is, depending on the particular code being obfuscated, a competent reverse engineer may well find it possible
understand still, or even be able to reorder the statements in such a way that they are closer the original program and clearer
to the human thought process. The positive token we take from the programatic analyser's blindness to these matters is that it
can be quite difficult to perform the reverse obfusction automatically.

\subsection{Method}

We have so far been talking about grouping and reordering of statements. To be clear, a statement here means any node in Python's abstract
grammar \cite{pyagrammar} labelled as a \texttt{stmt} node. This includes any expression, as well as things like function definitions or
assignments. Compound statments are also valid, including among others, if-else statements or for or while loops. With this constraint on
what we count as a statement, we can now say that we are able to reorder anything on the top level of a module or inside a suite in a
compound statement.

At this point we wish to learn how to reorder statements without breaking the program. In terms of our implementation, we should be able to
receive an ordered list of AST nodes representing statements and calculate somehow the other permutations of these statments which have the
same visible output.

Once we have code to produce these valid permutations, we can obfuscate or deobfuscate by our choice of permutation. An obfuscator would value
orderings holding less information and a deobfuscator would favour those with more. How we decide which orderings hold the information will be
a bridge to cross later on.

So now how do we calculate these valid permutations? Well our approach will be to mark each node with a number of pieces of information. For now
this will be done manually by the user, but we will discuss in the Automarking section ways in which we could automate this process. Using this
information we can cast out invalid permutations and assume any left are correct. The specific information we need to mark nodes with will be decided
as we walk through our constraints.

\subsubsection{Visibility}

The first constraint we will think about is the order of visible output. As we have already stated, we want the visible output of the program
to remain the same after any valid permutation.

Not all statements will be creating this output, for example a print statement or a call to a function that contains it will, but simple assignment
to a constant will not. We will call those that do produce visible output visible statements, and mark them as such. Actually if we are unsure whether
a statement is visible, it is safer to assume it is rather than the alternative. By doing this we will only introduce a new constraint rather than losing
one that may have been necessary for the correct running of the program.

Once the (possibly) visible statements have been marked we can easily see that they should not be reordered relative to each other or the visible output as a
whole will be altered. This is our first constraint.

\subsubsection{Breaking Statements}

Next we think about which statements should be executed and the effect of any breaking statements on the rest of the code.

By breaking statements we mean any statement that may break from the normal flow of the program. That is, if any statement in our list can
prevent the following statements from being executed by exiting the current block then it is a breaking statement.

These statements will be marked not with the fact that they can break program flow, but the type of break they may produce. The reason for this
is to allow easier implementation of the automarker. Currently the allowed types of breaking statement are those which may throw exceptions, return,
yield, break or continue. If any of these breaks are occur due to a statement in our list, the nodes following it will not be executed - the break can
only be caught higher up in the program.

After the user has marked these statements with the possible types of break they generate, we can start to think about how they affect our visible output.
For now we will only discuss break types other than yield.

Our main concern for a breaking statement is that for all breaks other than yield, the statement list will only be executed until the break. After this
nothing will be executed. This means statement before a break will need to stay before the break and any after will need to stay after if we wish it to be
executed in the same circumstances after the permutation.

Usually we would only care about applying this restriction to statements that are visible and indirectly to those that visible statements rely on, however
there are some cases where we would need to think about the other statements too. For instance a programmer may try to avoid an infinite loop in a function
by performing a quick check at the beginning and returning prematurely if the loop would be infinite. Whether or not this loop is visible, we do want to avoid
moving it before the possible break as we could then end up stuck in an infinite loop which will break the program.

There is more than one way to deal with this problem. One suggestion would to be to assume a statement that never terminates, or even one that runs for a long
time is a visible statement. In a way this is true as it does affect the user's experience, although it is less natural to think of as visible than a print or
a write to a file. This is probably a bad idea as it is often difficult to tell how long a statement will run for and leaves it up to a user to decide how long
is noticeable.

A better way we could deal with breaking statements is to decide no statement can move to the other side of a breaking statement, effectively partitioning the
statement list between each breaking statement. Now we can treat each partition as a separate problem, reordering them as if there are no breaking statements, as
long as the partitions are kept in order. As all statements that would otherwise have been executed are still executed, we should also expect execution time to
remain constant which is an ideal situation.

Now before we move on, we should take a look at the yield statement. This statement makes things difficult as any function containing this statement becomes a
generator function. Running the generator function actually returns an iterator. Each time we ask the iterator for the next item, the function is run until the
next yield or return statement. After this, if the iterator is asked to continue it will do so from just after the yield statement it left at, or throw an exception
if it left through a return.

We want to know if our previous decision about how to treat breaking statements still works in this case. Well all we need to know is that a visible statement will
still be run or not run after a valid permutation. Well if we partition the statements as described above then this is true. All the statements are already split
so that any between yield or return statements must stay there. Now all statements that would have been executed in a given iteration will still be executed.

This partitioning forms our second constraint on the permutations.

\subsubsection{Read/Write Dependencies}

Now we have the two simpler constraints out the way, we should look at the part which takes most of the processing when choosing valid permutations. This is
keeping read/write dependencies unbroken.

So far we have been trying to make sure the correct statements are always run. This alone does not ensure correctness of our permutation. Take for instance
the following:

\begin{lstlisting}
a = "hello"
print(a)    # Prints "hello"
a = "world"
print(a)    # Prints "world"
\end{lstlisting}

We see that lines 2 and 4 are identical (excluding the comments), yet they produce entirely different output. We want to make sure this doesn't happen in
our permutations. The way we achieve this is by ensuring that relevant state is kept identical when entering the each statement. If this is true then we can
be sure the statement will behave in the same way 

\section{Buuullllsshhhiiiitttt}

for each node in the list to build a list of every other node in the list that that
particular node is dependant on. More simply, if the effect of statement A could change the effect of statement B, then B depends on A. For internal
data such as variable assignments this means if A changes a memory region R that B accesses at some point in the future, then B depends on A.

Also we must look at statements that may perform external changes such as a print statement. An observer of the program would notice immediately
a change in order of these. Therefore all statements capable of change the world external to the program must remain in order relative to each other.
We can make sure of this by assuming any function with external effect depends on the IO function immediately previous to it.

When working out possible permutations we can now just search for options that neither add nor remove dependencies between the instructions.

\subsection{Looking for dependencies}

In calculating dependencies, we will start with the simplest possible list of statements. From here we will expand our reasoning slowly until
we can analyse a substantial portion of the Python language.

\subsubsection{Simplest Case}

Our simplest case would be one where none of the instructions depend on each other. We will ignore external output throughout as we already know
all we need to do is keep any such functions in order. We allow only assignments here as possible statements, and those that bind constants to
variables. The assignments should have no effect other than to add the constant into a new region of memory and assign the given name to
reference this location. This should model a Python program if all identifiers on the left hand side of the assignment are simple names.
(We also ignore the changes Python will make to certain data such as that returned by the \texttt{locals()} built-in function).

An example:

\begin{lstlisting}
a = 1
b = 2
c = []
c = 3
\end{lstlisting}

Here the instructions could move to any order. None of the assignments rely on the previous assignments of any other value to any other name. It's
worth looking at the two assignments of the same variable, \texttt{c}. If we reversed the order of these two statements we would finish the
statement list with either the local dictionary \texttt{\{a: 1, b: 2, c: []\}} or \texttt{\{a: 1, b: 2, c: 3\}}.

In fact this difference does not matter. This change is internal to the program and so not visible to an observer. As we have not specified
any future dependency on \texttt{c}, we can assume that nothing that can affect the observable behaviour of the program will use the value
of \texttt{c} - meaning the program remains correct.

\subsubsection{Slightly Harder}

Now we allow assignments of values that can either be constant or other local names. We will restrict all of our names to local scope as
we have not yet introduced the concept of scope to our model.

An example:

\begin{lstlisting}
a = 1
b = a
c = a
c = b
\end{lstlisting}

Now we can easily go through line by line and look at which names appear on the right of the assignment to gather a list of dependencies.

TODO - CONTINUE FROM HERE!-------------------------------------------------------------

Problem! What if we added one line to make:

\begin{lstlisting}
a = 1
b = a
d = e
c = a
c = b
\end{lstlisting}

Now we have a line that will cause error. 

\section{Opaque Predicates}

The second obfuscation technique we will implement will insert branches into the code that depend on
some opaque predicate \cite{taxobftrans}. This is a predicate that we as the obfuscator know the value of,
but an analyser will find it difficult to evaluate. The branches we will insert will jump either to our valid
program code, or another obfuscated version of our code or even some cleverly disguised garbage. The purpose of doing so being
that the increased number of branches in the program will confuse both a human reader and a programatic analyser.

To reverse the transformation, dead code analysis will be used. This means the analyser will seek to remove
branches where we know a program will always take a certain path. The success of this analysis
depends on both the strength of the analysis and the strength of the opaque predicate.

\section{Relocate or Implement me - TODO}

Both of the described transformations and their analysis will require careful testing. So far,
this has been done using Python's doc test library to write unit tests inside the docstrings
in each piece of code. It is likely a higher level approach will be more suitable later in
the project, testing a program's validity before and after obfuscations as well as the validity
of the analysis.

\section{Evaluation}

\subsection{How to Evaluate}

Evaluation of the final product will be difficult, this is mainly due to the lack of similar tools.
Comparison to other Python obfuscators is not viable as only the pyobfuscate tool is available to
use. While this does perform obfuscation on Python source, it does not operate on the same version of
the source or perform the same type of obfuscation. In fact most available obfuscators will deal with
layout transformation rather than control flow.

There are a few obfuscators that will deal with control flow obfuscation in other languages, for example
the Kava tool for Java is discussed in detail elsewhere \cite{taxobftrans}. We could attempt to compare these
tools with our own, however they tend to be written for much less flexible or dynamic languages than
Python (and hence are easier analysable). Indeed, it is one of the intentions of this project to assess the
extra difficulty a dynamic language such as Python adds when writing these tools, but to compare the final
product to one of these existing tools from such a different base would be misleading.

As well as comparing to similar tools, we have the option to assess an obfuscator's merit based on
its ability to create output that withstands reverse engineering or analysis. In fact this should be
a much better analysis of the product as we can say if the software is good at what it is designed to do, rather
than whether it is just better at performing the same function as another tool.

To aid in this form of evaluation we will need tools to analyse the obfuscated source code. There are plenty
mentioned in the background section, however these tend to be designed to optimise code or detect possible bugs and so
are not ideal.

The solution proposed for this project is to write tools for analysis along side the tools for obfuscation.
By doing this we can really target the analysis at detecting and reversing the particular transformations
that our obfuscator implements. Hopefully this should not create a great deal of extra work as the analysis
needed for reversing a transformation is often likely to be the same form of analysis performed to make that
transformation in the first place, so the work is already done.

Of course this is not a perfect solution. Ideally analytic tools would be written by someone else without
knowledge of the specific obfuscator. This would move closer to representing a reverse-engineer or analytic
tool in the real world, as hopefully neither would be able to ascertain the software used to perform the obfuscation
or the specific techniques used on specific code locations. In theory this should give the analytic tool an
advantage, lowering our expected potency for the obfuscator.

Another problem caused by the same author writing both sides of a competing set of software is that the competition
may not be so fierce as in the real world. Any design oversights in one half of the software will likely show up
in the other, rather than being exploited as would hopefully be the case with two free thinking developers. As a real
example, the Java decompiler Mocha will crash if it finds extra statements after a function has returned
\cite{hosemocha}[p4]. The bytecode obfuscator HoseMocha uses this to prevent Mocha from decompiling programs. Had these two
pieces of software been written by the same developer, one whose decompiler doesn't check for the unlikely case described,
then it is probable that this case would never have been thought of for the obfuscator - and HoseMocha would never have
been created.

We do not have the option of contracting a separate developer to help with this evaluation. The best solution to mitigate the effect
of this is to compare the development of the analysis tool to the available analytic tools discussed in the background section. Although
these have different primary functions, at least some functionality will overlap and allow a loose comparison. From this it
should be possible to determine if our analysis tool is sub-par or better than average and to judge the obfuscator
accordingly.

\subsection{What to Evaluate}

We have looked at how we can evaluate the software, and we know the tools we will use to do so. What we don't know yet is
the specific criteria that we will be evaluating. 

As we already know, the aim of obfuscation is to confuse a human reader or a programmatic analysis. In measuring the effectiveness
of a particular obfuscation we are actually attempting to quantify the additional confusion caused by it. This is not an easy task
as confusion is very subjective to a particular viewer or program.

Previous attempts to do this tend to use program complexity metrics to measure the impact of a particular obfuscation. Many fail to
actually quantify transformation benefit, preferring instead to give a rough estimation, although recently there have been improvements in
this area \cite{obfquant}.

Although these metrics can be used during the obfuscating and analysis, they are unlikely to help with evaluation as we are
not proposing any new transformations. Instead we are comparing the same transformation in an environment with looser rules than usual.
In this case, complexity metrics will be very similar if not the same and so not helpful to us.

A much more helpful approach will be to discuss freely (and using the metrics) the challenges faced in implementing these obfuscations
in Python as well as breaking them. This can be compared to other languages and implementations. Finally the comparative performances
of the obfuscator and analyser will be assessed on some example programs.

\section{Conclusions and Future Work}

TODO - this section, aaah

\bibliographystyle{plain}
\bibliography{final}

\end{document}
